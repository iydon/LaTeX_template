\documentclass[landscape,a0paper,fontscale=0.285]{baposter} % Adjust the font scale/size here

% Poster environment
\makeatletter
% TITLE PAGE
\def\@logoleft{\includegraphics[height = 4em]{logo-school-en.pdf}}
\def\@logoright{\includegraphics[height = 4em]{logo-department-en.pdf}}
\def\@title{HELLO WORLD!}
\def\@author{Iydon}
% COLOR
\def\@colorfrom{white}
\def\@colorto{white}
\def\@titlecolor{white}

\newcommand\logo[2]{\def\@logoleft{#1}\def\@logoright{#2}}
\newcommand\colorfromto[2]{\def\@colorfrom{#1}\def\@colorto{#2}}
\newcommand\titlecolor[1]{\def\@titlecolor{#1}}


\newenvironment{postech}
{\begin{poster}{
        headerborder = closed,          % Adds a border around the header of content boxes
        colspacing = 1em,               % Column spacing
        bgColorOne = white,             % Background color for the gradient on the left side of the poster
        bgColorTwo = white,             % Background color for the gradient on the right side of the poster
        borderColor = \@colorto,        % Border color
        headerColorOne = \@colorfrom,   % Background color for the header in the content boxes (left side)
        headerColorTwo = \@colorto,     % Background color for the header in the content boxes (right side)
        headerFontColor = \@titlecolor, % Text color for the header text in the content boxes
        boxColorOne = white,            % Background color of the content boxes
        textborder = roundedleft,       % Format of the border around content boxes, can be: none, bars, coils, triangles, rectangle, rounded, roundedsmall, roundedright or faded
        eyecatcher = true,              % Set to false for ignoring the left logo in the title and move the title left
        headerheight = 0.1\textheight,  % Height of the header
        headershape = roundedright,     % Specify the rounded corner in the content box headers, can be: rectangle, small-rounded, roundedright, roundedleft or rounded
        headerfont = \Large\bf\textsc,  % Large, bold and sans serif font in the headers of content boxes
        linewidth = 2pt                 % Width of the border lines around content boxes
        % textfont = {\setlength{\parindent}{1.5em}}, % Uncomment for paragraph indentation
    }
    %----------------------------------------------------------------------------------------
    %    TITLE SECTION
    %----------------------------------------------------------------------------------------
    %
    {\@logoleft} % First university/lab logo on the left
    {\bf\textsc{\@title}\vspace{0.5em}} % Poster title
    {\textsc{\@author}} % Author names and institution
    {\@logoright} % Second university/lab logo on the right
}
{\end{poster}}
\makeatother

% !Mode:: "TeX:UTF-8"

% ------- Packages -------
\usepackage{palatino}    % Palatino/Helvetica/TXTT and Greek replacements
\usepackage{lipsum}        % 生成测试文本
\usepackage{hologo}        % TeX符号
\usepackage{amsthm}        % Theorem
    \newtheorem{thm}{Theorem}[section]
    \newtheorem{cor}{Corollary}[section]
    \newtheorem{defn}{Definition}[section]
\usepackage{fontawesome}   % 图标
\usepackage{hyperref}      % 超链接
% \usepackage{PyLaTeX}       % https://github.com/Iydon/LaTeX_template/tree/master/PyLaTeX
%     \pylatexpycmd{python3} % Python cmd name
%     \pylatexdoit

% ------- Symbols -------
% !Mode:: "TeX:UTF-8"
\usepackage{xspace}
\usepackage{mathrsfs}

% Fette Kleinbuchstaben (fuer Vektoren)
\newcommand{\Va}{{\mathbf{a}}}
\newcommand{\Vb}{{\mathbf{b}}}
\newcommand{\Vc}{{\mathbf{c}}}
\newcommand{\Vd}{{\mathbf{d}}}
\newcommand{\Ve}{{\mathbf{e}}}
\newcommand{\Vf}{{\mathbf{f}}}
\newcommand{\Vg}{{\mathbf{g}}}
\newcommand{\Vh}{{\mathbf{h}}}
\newcommand{\Vi}{{\mathbf{i}}}
\newcommand{\Vj}{{\mathbf{j}}}
\newcommand{\Vk}{{\mathbf{k}}}
\newcommand{\Vl}{{\mathbf{l}}}
\newcommand{\Vm}{{\mathbf{m}}}
\newcommand{\Vn}{{\mathbf{n}}}
\newcommand{\Vo}{{\mathbf{o}}}
\newcommand{\Vp}{{\mathbf{p}}}
\newcommand{\Vq}{{\mathbf{q}}}
\newcommand{\Vr}{{\mathbf{r}}}
\newcommand{\Vs}{{\mathbf{s}}}
\newcommand{\Vt}{{\mathbf{t}}}
\newcommand{\Vu}{{\mathbf{u}}}
\newcommand{\Vv}{{\mathbf{v}}}
\newcommand{\Vw}{{\mathbf{w}}}
\newcommand{\Vx}{{\mathbf{x}}}
\newcommand{\Vy}{{\mathbf{y}}}
\newcommand{\Vz}{{\mathbf{z}}}

% Small bold math characters
\newcommand{\Ba}{{\boldsymbol{a}}}
\newcommand{\Bb}{{\boldsymbol{b}}}
\newcommand{\Bc}{{\boldsymbol{c}}}
\newcommand{\Bd}{{\boldsymbol{d}}}
\newcommand{\Be}{{\boldsymbol{e}}}
\newcommand{\Bf}{{\boldsymbol{f}}}
\newcommand{\Bg}{{\boldsymbol{g}}}
\newcommand{\Bh}{{\boldsymbol{h}}}
\newcommand{\Bi}{{\boldsymbol{i}}}
\newcommand{\Bj}{{\boldsymbol{j}}}
\newcommand{\Bk}{{\boldsymbol{k}}}
\newcommand{\Bl}{{\boldsymbol{l}}}
\newcommand{\Bm}{{\boldsymbol{m}}}
\newcommand{\Bn}{{\boldsymbol{n}}}
\newcommand{\Bo}{{\boldsymbol{o}}}
\newcommand{\Bp}{{\boldsymbol{p}}}
\newcommand{\Bq}{{\boldsymbol{q}}}
\newcommand{\Br}{{\boldsymbol{r}}}
\newcommand{\Bs}{{\boldsymbol{s}}}
\newcommand{\Bt}{{\boldsymbol{t}}}
\newcommand{\Bu}{{\boldsymbol{u}}}
\newcommand{\Bv}{{\boldsymbol{v}}}
\newcommand{\Bw}{{\boldsymbol{w}}}
\newcommand{\Bx}{{\boldsymbol{x}}}
\newcommand{\By}{{\boldsymbol{y}}}
\newcommand{\Bz}{{\boldsymbol{z}}}

% Small bold characters with hat on top
\newcommand{\VaH}{\widehat{\mathbf{a}}}
\newcommand{\VbH}{\widehat{\mathbf{b}}}
\newcommand{\VcH}{\widehat{\mathbf{c}}}
\newcommand{\VdH}{\widehat{\mathbf{d}}}
\newcommand{\VeH}{\widehat{\mathbf{e}}}
\newcommand{\VfH}{\widehat{\mathbf{f}}}
\newcommand{\VgH}{\widehat{\mathbf{g}}}
\newcommand{\VhH}{\widehat{\mathbf{h}}}
\newcommand{\ViH}{\widehat{\boldsymbol \imath}}
\newcommand{\VjH}{\widehat{\boldsymbol \jmath}}
\newcommand{\VkH}{\widehat{\mathbf{k}}}
\newcommand{\VlH}{\widehat{\mathbf{l}}}
\newcommand{\VmH}{\widehat{\mathbf{m}}}
\newcommand{\VnH}{\widehat{\mathbf{n}}}
\newcommand{\VoH}{\widehat{\mathbf{o}}}
\newcommand{\VpH}{\widehat{\mathbf{p}}}
\newcommand{\VqH}{\widehat{\mathbf{q}}}
\newcommand{\VrH}{\widehat{\mathbf{r}}}
\newcommand{\VsH}{\widehat{\mathbf{s}}}
\newcommand{\VtH}{\widehat{\mathbf{t}}}
\newcommand{\VuH}{\widehat{\mathbf{u}}}
\newcommand{\VvH}{\widehat{\mathbf{v}}}
\newcommand{\VwH}{\widehat{\mathbf{w}}}
\newcommand{\VxH}{\widehat{\mathbf{x}}}
\newcommand{\VyH}{\widehat{\mathbf{y}}}
\newcommand{\VzH}{\widehat{\mathbf{z}}}


% ------- Command and Definition -------
% 正文摘要和控制页摘要名字修改
\def\abstractname{Abstract}
\def\sheetsummaryname{Summary}

% ------- Information -------
\mcmsetup{CTeX = false,         % 使用 CTeX 套装时,设置为 true
        tcn = 0710,             % 队伍控制号码
        problem = A,            % 选题
        sheet = true,           % 摘要页
        titleinsheet = true,    % 摘要页中的标题
        keywordsinsheet = true, % 摘要页中的关键字
        titlepage = true,       % 标题页
        abstract = true,        % 标题页中的摘要
        XeTeX = true,           % XeTeX编译
}
\title{The \LaTeX{} Template for MCM Version \MCMversion}
\author{Iydon, Svégio, Xylocis}
\date{\today} 

\begin{document}

\begin{filecontents*}{\jobname.txt}
:reference{
┌─┐
│ │
└─┘
}
:layout{
┌─┐┌─┐┌───┐
└1┘└2┘|   |
┌─┐┌─┐└3──┘
| || |┌───┐
└4┘└5┘└6──┘
┌─┐┌───┐┌─┐
└7┘└8──┘└9┘
}
:name{
1: Objectives
2: Introduction
3: Results 1
4: Materials \& Methods
5: Results 2
6: Conclusion
7: References
8: Future Research
9: Contact Information
}
\end{filecontents*}
\immediate\write18{python3\space extract.py\space\jobname.tex}

\newcommand\posteri{%

Donec non nisl a \textbf{arcu consequat} varius. Sed suscipit cursus luctus. Nulla sit amet elit augue. Curabitur scelerisque mollis dolor, quis blandit lorem condimentum at. Pellentesque sed nibh vel \textbf{dolor} sagittis semper.

\begin{enumerate}\compresslist
\item Feugiat vitae elit
\item bibendum ante sed lacinia eros in
\item Curabitur scelerisque arcu consequat varius
\item Dapibus nulla id purus consectetur
\item Fringilla integer
\end{enumerate}

\vspace{0.3em} % When there are two boxes, some whitespace may need to be added if the one on the right has more content
}



\newcommand\posterii{%

Aliquam non lacus dolor, \textit{a aliquam quam}. Cum sociis natoque penatibus et magnis dis parturient montes, nascetur ridiculus mus. Nulla in nibh mauris. Donec vel ligula nisi, a lacinia arcu. Sed mi dui, malesuada vel consectetur et, egestas porta nisi. Sed eleifend pharetra dolor, et dapibus est vulputate eu. \textbf{Integer faucibus elementum felis vitae fringilla.} In hac habitasse platea dictumst. Duis tristique rutrum nisl, nec vulputate elit porta ut. Donec sodales sollicitudin turpis sed convallis. Etiam mauris ligula, blandit adipiscing condimentum eu, dapibus pellentesque risus.
}



\newcommand\posteriii{%

\begin{multicols}{2}
\vspace{1em}
\begin{center}
\includegraphics[width=0.8\linewidth]{placeholder}
\captionof{figure}{Figure caption}
\end{center}

Aliquam auctor, metus id ultrices porta, risus enim cursus sapien, quis iaculis sapien tortor sed odio. Mauris ante orci, euismod vitae tincidunt eu, porta ut neque. Aenean sapien est, viverra vel lacinia nec, venenatis eu nulla. Maecenas ut nunc nibh, et tempus libero. Aenean vitae risus ante. Pellentesque condimentum dui. Etiam sagittis purus non tellus tempor volutpat. Donec et dui non massa tristique adipiscing.
\end{multicols}

%------------------------------------------------

\begin{multicols}{2}
\vspace{1em}
Sed fringilla tempus hendrerit. Vestibulum ante ipsum primis in faucibus orci luctus et ultrices posuere cubilia Curae; Etiam ut elit sit amet metus lobortis consequat sit amet in libero. Lorem ipsum dolor sit amet, consectetur adipiscing elit. Phasellus vel sem magna. Nunc at convallis urna. isus ante. Pellentesque condimentum dui. Etiam sagittis purus non tellus tempor volutpat. Donec et dui non massa tristique adipiscing. Quisque vestibulum eros eu.

\begin{center}
\includegraphics[width=0.8\linewidth]{placeholder}
\captionof{figure}{Figure caption}
\end{center}

\end{multicols}
}



\newcommand\postervii{%

\renewcommand{\section}[2]{\vskip 0.05em} % Get rid of the default "References" section title
\nocite{*} % Insert publications even if they are not cited in the poster
\small{ % Reduce the font size in this block
\bibliographystyle{unsrt}
\bibliography{sample} % Use sample.bib as the bibliography file
}}



\newcommand\posterviii{%

\begin{multicols}{2}
Integer sed lectus vel mauris euismod suscipit. Praesent a est a est ultricies pellentesque. Donec tincidunt, nunc in feugiat varius, lectus lectus auctor lorem, egestas molestie risus erat ut nibh.

Maecenas viverra ligula a risus blandit vel tincidunt est adipiscing. Suspendisse mollis iaculis sem, in \emph{imperdiet} orci porta vitae. Quisque id dui sed ante sollicitudin sagittis.
\end{multicols}
}



\newcommand\posterix{%

\begin{description}\compresslist
\item[Web] www.university.edu/smithlab
\item[Email] john@smith.com
\item[Phone] +1 (000) 111 1111
\end{description}
}



\newcommand\postervi{%

\begin{multicols}{2}

\tikzstyle{decision} = [diamond, draw, fill=blue!20, text width=4.5em, text badly centered, node distance=2cm, inner sep=0pt]
\tikzstyle{block} = [rectangle, draw, fill=blue!20, text width=5em, text centered, rounded corners, minimum height=4em]
\tikzstyle{line} = [draw, -latex']
\tikzstyle{cloud} = [draw, ellipse, fill=red!20, node distance=3cm, minimum height=2em]

\begin{tikzpicture}[node distance = 2cm, auto]
\node [block] (init) {Initialize Model};
\node [cloud, left of=init] (Start) {Start};
\node [cloud, right of=init] (Start2) {Start Two};
\node [block, below of=init] (init2) {Initialize Two};
\node [decision, below of=init2] (End) {End};
\path [line] (init) -- (init2);
\path [line] (init2) -- (End);
\path [line, dashed] (Start) -- (init);
\path [line, dashed] (Start2) -- (init);
\path [line, dashed] (Start2) |- (init2);
\end{tikzpicture}

%------------------------------------------------

\begin{itemize}\compresslist
\item Pellentesque eget orci eros. Fusce ultricies, tellus et pellentesque fringilla, ante massa luctus libero, quis tristique purus urna nec nibh. Phasellus fermentum rutrum elementum. Nam quis justo lectus.
\item Vestibulum sem ante, hendrerit a gravida ac, blandit quis magna.
\item Donec sem metus, facilisis at condimentum eget, vehicula ut massa. Morbi consequat, diam sed convallis tincidunt, arcu nunc.
\item Nunc at convallis urna. isus ante. Pellentesque condimentum dui. Etiam sagittis purus non tellus tempor volutpat. Donec et dui non massa tristique adipiscing.
\end{itemize}

\end{multicols}
}



\newcommand\posteriv{%

The following materials were required to complete the research:

\begin{itemize}\compresslist
\item Curabitur pellentesque dignissim
\item Eu facilisis est tempus quis
\item Duis porta consequat lorem
\item Eu facilisis est tempus quis
\end{itemize}

The following equations were used for statistical analysis:

\begin{equation}
\cos^3 \theta =\frac{1}{4}\cos\theta+\frac{3}{4}\cos 3\theta
\label{eq:refname}
\end{equation}\

\begin{equation}
E = mc^{2}
\label{eqn:Einstein}
\end{equation}

Phasellus imperdiet, tortor vitae congue bibendum, felis enim sagittis lorem, et volutpat ante orci sagittis mi. Morbi rutrum laoreet semper. Morbi accumsan enim nec tortor consectetur non commodo nisi sollicitudin. Proin sollicitudin. Pellentesque eget orci eros. Fusce ultricies, tellus et pellentesque fringilla, ante massa luctus libero, quis tristique purus urna nec nibh.
}



\newcommand\posterv{%

Donec faucibus purus at tortor egestas eu fermentum dolor facilisis. Maecenas tempor dui eu neque fringilla rutrum. Mauris \emph{lobortis} nisl accumsan.

\begin{center}
\begin{tabular}{l l l}
\toprule
\textbf{Treatments} & \textbf{Response 1} & \textbf{Response 2}\\
\midrule
Treatment 1 & 0.0003262 & 0.562 \\
Treatment 2 & 0.0015681 & 0.910 \\
Treatment 3 & 0.0009271 & 0.296 \\
\bottomrule
\end{tabular}
\captionof{table}{Table caption}
\end{center}

Nulla ut porttitor enim. Suspendisse venenatis dui eget eros gravida tempor. Mauris feugiat elit et augue placerat ultrices. Morbi accumsan enim nec tortor consectetur non commodo.

\begin{center}
\begin{tabular}{l l l}
\toprule
\textbf{Treatments} & \textbf{Response 1} & \textbf{Response 2}\\
\midrule
Treatment 1 & 0.0003262 & 0.562 \\
Treatment 2 & 0.0015681 & 0.910 \\
Treatment 3 & 0.0009271 & 0.296 \\
\bottomrule
\end{tabular}
\captionof{table}{Table caption}
\end{center}
}

\input{postent}

\end{document}
