\documentclass{ctexart}
\usepackage{hyperref}
\hypersetup{
    pdftitle={MathematicalModeling},
    pdfsubject={MA206数学建模},
    pdfauthor={Iydon Leong},
    pdfkeywords={KEYWORDS1, KEYWORDS2},
    pdfcreator={SUSTeX},
    pdfproducer={SUSTeX 0.1},
}
\usepackage[table,xcdraw]{xcolor}
\usepackage{threeparttable}
\usepackage{tabu}
\newcommand\Introduction{
    理论课(含实践),3学分,4学时/周。
    先修课程:常微分方程(MA201/MA201a/MA201b)。
    本课程是高强度介绍数学建模,主要使用图形、数值、符号计算和数学写作技巧描述和探究实际数据和现象。
    重点在运用微积分,高等代数的知识来研究和分析应用型的模型和问题,特别是物理、生态、环境、医学、管理、经济、信息技术等领域的一些典型实例,在传授知识的同时,通过典型建模实例的分析和参加建模实践活动,培养和增强学生自学能力、创新素质。
    Lecture, 3 credits, 3 hours per week. Pre-requisites:Ordinary Differential Equations (MA201/MA201a/MA201b).
    This course is an intensive introduction to mathematical modeling using graphical, numerical, symbolic, and verbal techniques to describe and explore real-world data and phenomena.
    Emphasis is on the use of calculus and linear algebra to investigate and analyze prototype model problems and interesting questions in science, physics, technology, medicine, economics and information science.
}
\begin{document}
\begin{table}[t]
\centering
    \begin{tabu} to \linewidth{|
    >{\columncolor[HTML]{9AFF99}}c |c|c|c|}
        \hline
        \textbf{学年学期} & 2018-2019-2 & \cellcolor[HTML]{9AFF99}\textbf{开课院系} & 数学系 \\ \hline
        \textbf{课程号} & MA206 & \cellcolor[HTML]{9AFF99}\textbf{课程名称} & 数学建模 \\ \hline
        \textbf{学分} & 3 & \cellcolor[HTML]{9AFF99}\textbf{总学时} & 64 \\ \hline
        \textbf{课程类别} & 理论课(含实践) & \cellcolor[HTML]{9AFF99}\textbf{课程性质} & 专业选修课 \\ \hline
        \textbf{课程属性} & 选修 & \cellcolor[HTML]{9AFF99}\textbf{状态} & 启用 \\ \hline
        \textbf{课程简介} & \multicolumn{3}{c|}{理论课(含实践),3学分,4学时/周。$^a$} \\ \hline
        \textbf{先选先修条件} & \multicolumn{3}{c|}{(常微分方程A 或者 常微分方程B 或者 常微分方程)} \\ \hline
    \end{tabu}
    \begin{itemize}
        \footnotesize
        \item[$^a$] \Introduction
    \end{itemize}
\end{table}
\end{document} 