%!TEX program = xelatex
\documentclass[openany]{ctexbook}
\usepackage[margin=1in]{geometry}
\usepackage{lipsum}
\usepackage{fontspec}
    \setmainfont{Times New Roman}
\usepackage{tcolorbox}
    \newenvironment{example}{\begin{tcolorbox}[title=Example]}{\end{tcolorbox}}
\usepackage{enumerate}
\begin{document}
\tableofcontents\clearpage


\chapter{小节划分}
    \section*{Summary}
    \lipsum[1]

    \section{Restatement of the Problem}
    引言应该包括以下内容:
    \begin{itemize}
    	\item 对赛题的解读.
    	\item 对现有研究成果的综述与评价.
    	\item 对解题思路和主要方法的简要介绍.
    \end{itemize}
    参赛小组首先要用自己的语言重述赛题, 明确解题目标, 并澄清原题叙述上可能出现的模糊概念.
    \begin{example}
        There are at least two notions of where the sweet spot should be --- an impact location on the bat that either
        \begin{itemize}
        	\item minimizes the discomfort to the hands, or
        	\item maximizes the outgoing velocity of the ball.
        \end{itemize}
        We focus exclusively on the second definition.
    \end{example}
    即便是已经表述得很精确的概念, 仍可以给出更有利于解题的解读方式.
    \begin{example}
        We interpret the error of $\pm 2^\circ$ as a normal distribution, $\ldots$ with standard deviation of $1^\circ$.
    \end{example}


    \section{Assumptions}
    合理的数学模型应基于合理的假设, 所以在描述模型之前, 参赛小组应该将模型设计所用的假设条件一一列出并解释清楚. 此外, 还应该对建模的初衷和动机适当地加以讨论. 参赛小组在论文中都应该明确列出所有用到的假设条件, 并解释其合理性.
    \begin{example}
        \textbf{Criminal's movement is unconstrained}. Because of the difficulty of finding real-world distance data, we invoke the ``Manhattan assumption'': There are enough streets and sidewalks in a sufficiently grid-like pattern that movements along real-world movement routes is the same as ``straight-line'' movement in a space discretized into city blocks $\ldots$
    \end{example}

    \section{Justification of Our Approach}
    \lipsum[4]

    \section{The Model}
    设计的模型能够解决问题才是最重要的. \textbf{在所有能解决问题的模型中, 最简单的模型也许就是最好的模型.}
        \subsection{系列模型}
        设计模型时, 可以尝试从简单模型开始, 逐步加工, 修改及完善, 一次比一次更接近实际, 最终得到能完满地解答赛题的模型.
        \begin{example}
            \begin{enumerate}[(1)]
                \item 首先建立模型: 常温模型. 这个模型假设全球温度不变, 冰盖的融化速度不变及海洋水量不变.
                \item 接着建立模型: 变温模型. 这个模型假设全球温度在不断变化.
                \item 最后建立模型: 气候变暖下的海洋水量模型. 这个模型将前面建立的模型中所忽略的问题考虑进来, 包括南北两半球的相对海洋水面面积.
            \end{enumerate}
        \end{example}

        \subsection{普遍模型}
        不要只针对赛题给出的参数值设计模型. 高水平的论文通常会把赛题看成是一个普遍问题的特例, 首先探讨普遍问题的求解, 然后再对赛题的这一特例给出具体解答.
        \begin{example}
            We have produced a general algorithm to solve this type of problem, but for out problem a relationship exists that greatly simplified the algorithm
        \end{example}

    \section{Testing the Model}
    参赛小组应该对所建立的数学模型进行敏感性分析和稳定性测试, 是模型更具有说服力. 模型通常会用到一些参数(例如, 交通建模问题可能会用到平均速度这一参数), 在结论中应该讨论这些参数值的轻微变化对模型及结论产生的影响.

    \section{Results}
    在描述结论时, 应设法使读者认同论文给出的解答, 尽管该解答不一定是最好的. 测试时应极可能使用真实的数据, 避免因为人造数据而引起读者对结论的怀疑.\par
    在描述结果时一定要给出足够的信息, 是的如果有必要, 读者自己也能得到相同的结果. 如果参赛小组自己编写了程序代码, 则应将程序运行的算法描述清楚.

    \section{Strengths and Weaknesses}
    由于时间的限制, 几乎每篇论文给出的模型和解答都会存在这样或那样的缺陷. 评委肯定也会发现这些缺陷, 所以在写结论时应该明确指出这些缺陷, 标明参赛小组不但知道模型所含的缺陷, 而且也思考过如何改进和修补. 论文缺陷在前面描述模型时可能已经提到过, 在这里应该再次指出来.\par
    写论文时也应该明确指出, 假如有充足的时间和计算资源, 参赛小组将能够解决问题.

    \section*{References}
    \lipsum[10]


\chapter{写作提示}
    \begin{itemize}
        \item 将论文划分小节时, 应避免在小节中出现大段的文字叙述.
        \item 首次定义的概念, 应该用黑体或斜体书写. 但在突出重点的前提下应可能少用黑体或斜体.
        \item 重要的数学公式应另起新行单独列出.
        \item 建模所用的假设条件以及所有可以用列表方式表述的内容, 应用列表的方式逐条陈列出来.
        \item 在使用图表的时候要给每个图表加上简单明确的文字说明.
    \end{itemize}
\end{document}
