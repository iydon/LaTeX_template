%!TEX program = xelatex
\documentclass[openany]{ctexbook}
\usepackage[margin=1in]{geometry}
\usepackage{lipsum}
\begin{document}
\chapter{小节划分}
    \section*{Summary}
    \lipsum[1]

    \section{Restatement of the Problem}
    \lipsum[2]

    \section{Assumptions}
    \lipsum[3]

    \section{Justification of Our Approach}
    \lipsum[4]

    \section{The Model}
        \subsection{lipsum}
        \lipsum[5]

        \subsection{lipsum}
        \lipsum[6]

    \section{Testing the Model}
    \lipsum[7]

    \section{Results}
    \lipsum[8]

    \section{Strengths and Weaknesses}
    \lipsum[9]

    \section*{References}
    \lipsum[10]


\chapter{写作提示}
    \begin{itemize}
        \item 将论文划分小节时, 应避免在小节中出现大段的文字叙述.
        \item 首次定义的概念, 应该用黑体或斜体书写. 但在突出重点的前提下应可能少用黑体或斜体.
        \item 重要的数学公式应另起新行单独列出.
        \item 建模所用的假设条件以及所有可以用列表方式表述的内容, 应用列表的方式逐条陈列出来.
        \item 在使用图表的时候要给每个图表加上简单明确的文字说明.
    \end{itemize}
\end{document}
